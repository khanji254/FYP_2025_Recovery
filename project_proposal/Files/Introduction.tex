\section{Introduction}\label{sec:introduction}

\subsection{Background of Study}

Rocket recovery systems represent the most critical subsystem for preserving flight data and hardware in high-power rocketry missions. The evolution of recovery systems has progressed from simple single deployment methods to sophisticated dual deployment architectures. Single deployment systems release a parachute at apogee, presenting risks of high drift radius and unpredictable landing zones. In contrast, dual deployment systems utilize a drogue parachute at apogee for stability, followed by a main parachute deployment at lower altitude for controlled soft landing, establishing the standard for modern high-power rocketry operations.

The Nakuja Project N4 mission builds upon previous experiences from the N3.5 mission, which employed a basic single-deployment piston-based system. The current N4 design implements dual deployment architecture but faces significant challenges in deployment reliability and telemetry signal integrity. Previous missions demonstrated critical vulnerabilities in both mechanical deployment mechanisms and communication systems, necessitating a comprehensive redesign approach.

\subsection{Problem Statement}

The N4 mission's recovery system faces critical reliability challenges including telemetry range limitations causing signal degradation during critical flight phases, uncontrolled ignition system temperatures exceeding 1000 degrees Celsius that create unreliable single-use deployment mechanisms, and absence of real-time pre-flight monitoring systems for battery voltage and igniter continuity, collectively threatening mission success through potential vehicle loss, deployment failures, and inability to validate system readiness before launch.

\subsection{Aim}

To design, fabricate, and validate a reliable dual deployment recovery system for the N4 rocket with extended telemetry range and comprehensive pre-flight monitoring capabilities.

\subsection{Specific Objectives}

The following four objectives address the identified challenges:

\begin{enumerate}
    \item \textbf{Long-Range Telemetry Implementation:} Implement Digi XBee-PRO 900HP telemetry system for robust long-range data transmission exceeding 5km with less than 5\% packet loss, enabling real-time monitoring throughout the flight envelope.
    
    \item \textbf{Pyrotechnic Deployment System:} Design and validate PWM-controlled ignition system for regulating nichrome wire temperature, coupled with sealed explosion cap design for consistent and reliable parachute deployment at apogee and main deployment altitudes.
    
    \item \textbf{Integrated Mechanical and Monitoring System:} Develop ruggedized 3D-printed PETG avionics holder with vibration isolation and battery retention, integrated with comprehensive feedback monitoring for battery voltage, deployment pin continuity, and system health diagnostics with launch inhibit functionality.
    
    \item \textbf{System Validation through Testing:} Conduct comprehensive pop tests to validate explosion cap deployment force and thermal containment, and execute range tests to verify telemetry performance at distances exceeding 5km under field conditions.
\end{enumerate}

\subsection{Justification}

The development of a reliable recovery system is paramount to mission success. Loss of the rocket vehicle results not only in hardware loss but also complete loss of flight data critical for mission analysis and future design improvements. The proposed improvements address documented failure modes from previous missions and implement industry-standard practices adapted for student rocketry applications.

The Nakuja Project represents a significant investment in student engineering education and practical aerospace systems development. Ensuring mission success through robust recovery systems validates this investment and provides valuable learning opportunities for future engineering professionals.
