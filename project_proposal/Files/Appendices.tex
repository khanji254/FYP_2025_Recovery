\appendix

\section{Project Budget}
\label{appendix:budget}

\begin{table}[htbp]
\centering
\caption{N4 Recovery System Budget Breakdown}
\label{tab:budget}
\footnotesize
\begin{tabular}{@{}llccc@{}}
\toprule
\textbf{Item} & \textbf{Description} & \textbf{Qty} & \textbf{Cost/Unit (KES)} & \textbf{Total (KES)} \\
\midrule
Telemetry Modules* & Digi XBee-PRO 900HP & 2 & 15,000 & 30,000 \\
Antennas & 900MHz High Gain Antenna & 2 & 2,000 & 4,000 \\
Copper Cladding & PCB Board (15 x 20 cm) & 3 & 200 & 600 \\
Microcontroller & ESP32 Development Board & 1 & 1,800 & 1,800 \\
Sensors & BMP180 + MPU6050 Module & 1 & 4,710 & 4,710 \\
Mosfets & IRL244N (Pack of 5) & 1 & 1,000 & 1,000 \\
\midrule
Pyrotechnics* & Black/Crimson Powder & 1 & 1,600 & 1,600 \\
Mechanical Parts* & Threaded Rods \& Hardware & 1 & 1,000 & 1,000 \\
Filament & PLA (1kg Roll) & 1 & 3,000 & 3,000 \\
Batteries & Lithium-Ion Cells (18650) & 4 & 500 & 2,000 \\
\midrule
\multicolumn{4}{l}{\textbf{Total (With Purchased)}} & \textbf{49,710} \\
\multicolumn{4}{l}{\textbf{Total (Without Purchased)}} & \textbf{13,110} \\
\bottomrule
\end{tabular}
\end{table}

*Components already available from Nakuja Project inventory. Out-of-pocket expenses: KES 13,110. Funding sources: Personal contribution (KES 7,000), Nakuja Project allocation (KES 6,110), university facilities for fabrication (in-kind).

\section{Project Timeline}
\label{appendix:timeline}

\begin{figure}[h]
\centering
\includegraphics[width=0.95\textwidth]{images/Timeplan.PNG}
\caption{15-week project timeline showing task dependencies and critical path through design, fabrication, and testing phases}
\label{fig:timeline_gantt}
\end{figure}

Implementation follows a 15-week schedule with critical milestones:
\begin{itemize}
    \item \textbf{Weeks 1-4:} Proposal presentation, continuous presentation, literature review and design phase
    \item \textbf{Weeks 5-8:} Mechanical, electrical, and control design with interim report submission
    \item \textbf{Weeks 9-11:} Material requisition, fabrication, and programming
    \item \textbf{Weeks 12-14:} Testing, demonstration, and final report preparation
    \item \textbf{Week 15:} Final report submission and project closeout
\end{itemize}

Critical path: PCB fabrication (2-week lead time), component procurement (3-week buffer). Risk mitigation: backup components ordered Week 4, 2-week weather contingency for launch operations.

\section{Antenna and RF Design Analysis}
\label{appendix:antenna_design}

\subsection{Null Space Signal Pattern}

The XBee-PRO 900HP antenna radiation pattern exhibits directional characteristics with null spaces in certain directions. Understanding these null spaces is critical for ground station placement and receiver diversity architecture.

\begin{figure}[htbp]
\centering
\includegraphics[width=0.8\textwidth]{images/donut_pattern.png}
\caption{XBee-PRO 900HP antenna radiation pattern showing signal strength distribution and null space regions. The pattern demonstrates approximate omnidirectional coverage in the horizontal plane with defined null zones in vertical extremes}
\label{fig:antenna_null_pattern}
\end{figure}

The null space analysis informs receiver antenna placement strategy, requiring minimum two-antenna diversity system at ground station to maintain continuous signal lock during all phases of the mission profile.

\subsection{RSSI Signal Decay Characterization}

Figure~\ref{fig:rssi_decay} presents empirical Received Signal Strength Indicator (RSSI) measurements at various distances from the airborne transmitter. This data validates the 5+km effective range specification for the XBee-PRO 900HP system and informs link budget calculations for launch site selection.

\begin{figure}[htbp]
\centering
\includegraphics[width=0.8\textwidth]{images/Beacon_RSSI_Decay_Plot.PNG}
\caption{RSSI signal decay plot showing measured signal strength attenuation with distance. Data confirms 5+km operational range with sufficient link margin (minimum -100 dBm receiver sensitivity) throughout nominal flight profile}
\label{fig:rssi_decay}
\end{figure}

The empirical RSSI decay curve provides quantitative validation for the telemetry system design and supports risk mitigation planning for boundary conditions (high-altitude drift, magnetic anomaly zones, etc.).

\section{Previous N4 Design References}
\label{appendix:previous_designs}

\begin{figure}[htbp]
\centering
\includegraphics[width=0.7\textwidth]{images/Nakuja_N3_Rocket.PNG}
\caption{Nakuja N3 rocket showing avionics bay configuration (reference design for N4 dual deployment architecture)}
\label{fig:n3_reference}
\end{figure}

Previous Nakuja missions (N1, N2, N3, N3.5) provide critical design heritage and lessons learned. The N3 single-deployment system identified key improvements needed for the N4 dual-deployment architecture, particularly in telemetry range extension (from 2km to 5+km) and pyrotechnic deployment reliability (see Figures~\ref{fig:current_circuit} and \ref{fig:current_fc_holder}).

\begin{figure}[htbp]
\centering
\includegraphics[width=0.75\textwidth]{images/Current_Circuit_Design.png}
\caption{Current N4 flight computer circuit design showing ESP32 microcontroller integration, XBee telemetry module, sensor interfaces (BMP180/MPU6050), MOSFET-based PWM ignition drivers, and power management circuitry}
\label{fig:current_circuit}
\end{figure}

\begin{figure}[htbp]
\centering
\includegraphics[width=0.6\textwidth]{images/Current_FC_Holder.png}
\caption{Current flight computer holder design featuring 3D-printed PETG construction with integrated battery retention mechanism, vibration isolation mounts, and modular component placement for easy assembly and maintenance}
\label{fig:current_fc_holder}
\end{figure}

The current design iterations (Figures~\ref{fig:current_circuit} and \ref{fig:current_fc_holder}) represent the latest improvements incorporating lessons from previous test campaigns, with enhanced vibration isolation and optimized circuit layout for reliability.

\section{Nakuja Project Resources}
\label{appendix:nakuja_resources}

\textbf{Project Website}: \url{https://nakujaproject.com/}

\textbf{GitHub Repository}: \url{https://github.com/nakujaproject}

The Nakuja Project maintains open-source repositories for:
\begin{itemize}
    \item Flight computer firmware (ESP32-based)
    \item Ground station telemetry software
    \item PCB design files (KiCAD format)
    \item Simulation and trajectory analysis tools
    \item Recovery system control algorithms
\end{itemize}

All documentation is available under open-source licenses to support the student rocketry community.
