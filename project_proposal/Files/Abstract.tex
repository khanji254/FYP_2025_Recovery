\section*{Abstract}
\addcontentsline{toc}{section}{Abstract}

Automated coffee berry sorting is an essential process for ensuring quality consistency and improving the economic value of coffee produce. Previous student-developed prototypes have demonstrated the feasibility of using image-based detection and air-ejection mechanisms for automated sorting. However, limitations related to motion blur, processing delays, and limited computational capacity have restricted real-time performance and throughput.

This project focuses on the modification and improvement of an existing coffee berry sorting machine model by addressing challenges associated with real-time image acquisition, detection, and classification. The study explores optimized computer vision algorithms, improved frame processing strategies, and alternative processing architectures to enhance system efficiency. A co-processing architecture alongside the Raspberry Pi 3 will be introduced to overcome computational bottlenecks, with task division between high-level control and time-critical processing operations.

Performance evaluation is conducted based on sorting accuracy, processing speed, and overall throughput. The improved model aims to provide a cost-effective, reliable, and scalable solution suitable for small-scale coffee processing applications while contributing to research in embedded vision and agricultural automation.
