\section{Expected Outcomes}
\label{sec:outcomes}

\subsection{Technical Performance}

\begin{itemize}
    \item \textbf{Sorting Accuracy}: Achievement of $>$90\% classification accuracy, up from current 75-80\% baseline \cite{color_based_sorting}
    \item \textbf{Throughput Capacity}: 150-200 kg/hr processing capacity, representing 2-2.5x improvement over current 50-75 kg/hr \cite{sorting_system_metrics}
    \item \textbf{False Accept Rate}: Reduction to $<$8\%, minimizing defects in accepted product stream
    \item \textbf{False Reject Rate}: Maintained at $<$12\%, optimizing yield while ensuring quality standards
    \item \textbf{System Reliability}: $>$95\% uptime during 4-hour continuous operation cycles, reducing maintenance interventions
    \item \textbf{Mechanical Robustness}: Elimination of belt tracking issues through crowned pulley design and spring tensioning \cite{belt_tracking_methods}
    \item \textbf{Sensor Performance}: Consistent color detection across variable lighting conditions via adaptive thresholding \cite{adaptive_thresholding}
\end{itemize}

\subsection{Economic Impact}

The enhanced sorting system will deliver measurable economic benefits:

\begin{itemize}
    \item \textbf{Labor Cost Reduction}: Reduced manual re-sorting requirements (50-70\% reduction in sorting labor hours)
    \item \textbf{Product Quality Improvement}: Higher proportion of premium-grade coffee through consistent defect removal
    \item \textbf{Processing Efficiency}: Elimination of throughput bottleneck in processing chain
    \item \textbf{Maintenance Cost Reduction}: Improved mechanical reliability reducing unplanned downtime and repair costs
    \item \textbf{Return on Investment}: Estimated 18-24 month payback period based on labor savings and quality improvements
\end{itemize}

\subsection{Project Timeline}

16-week implementation schedule: System Analysis and Design (Weeks 1-3), Component Procurement (Weeks 2-4), Mechanical Modifications (Weeks 4-7), Sensor and Control System Integration (Weeks 7-10), Component Testing (Weeks 10-12), System Integration and Calibration (Weeks 12-14), Performance Validation (Weeks 14-16). Detailed timeline provided in Appendix B.

Critical path activities include conveyor system modifications, sensor array fabrication and calibration, and control algorithm development and tuning.

\subsection{Knowledge Contribution}

Project documentation will provide:
\begin{itemize}
    \item Validated design specifications for mid-scale coffee berry sorting systems
    \item Sensor calibration procedures for RGB-based ripeness detection under African agricultural facility conditions
    \item Performance characterization data for solenoid-driven mechanical actuator sorting mechanisms
    \item Cost-benefit analysis comparing semi-automated sorting system improvements versus manual methods
    \item Troubleshooting guide for common failure modes in conveyor-based agricultural sorting systems
    \item Replicable design methodology applicable to other fruit and vegetable sorting challenges
\end{itemize}

These outputs will benefit agricultural engineering students, coffee processing cooperatives, and small-scale agricultural equipment manufacturers across Kenya and East Africa.

\subsection{Educational Value}

The project provides hands-on learning experiences in:
\begin{itemize}
    \item Mechatronic system integration combining mechanical, electrical, and software engineering
    \item Agricultural automation technology addressing real-world processing challenges
    \item Sensor technology application and signal processing algorithm development
    \item Performance optimization through systematic testing and parameter tuning
    \item Economic analysis and engineering design decision-making
\end{itemize}
