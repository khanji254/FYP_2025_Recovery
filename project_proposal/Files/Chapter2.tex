\section{Literature Review}
\label{sec:review}

\subsection{Recovery Systems in Student Rocketry}

Recovery systems preserve flight hardware and recorded data \cite{nakka_recovery}. Student rocketry projects face budget constraints requiring innovative, cost-effective solutions \cite{student_rocket_design}. Dual-deployment architectures use staged recovery: drogue chute at apogee for stability, main parachute at lower altitude for soft landing \cite{dual_deploy_analysis, parachute_design}.

\begin{figure}[h]
\centering
\fbox{\parbox{0.8\textwidth}{\centering\vspace{3cm}[Placeholder: Dual Deployment Recovery System Diagram]\vspace{3cm}}}
\caption{Dual deployment recovery system architecture showing drogue and main parachute deployment sequence}
\label{fig:dual_deploy}
\end{figure}

The Nakuja Project's progression from N3.5 single-deployment to N4 dual-deployment architecture reflects industry evolution \cite{nakuja_website}. Previous missions identified critical improvements needed in mechanical reliability and telemetry range.

\subsection{Telemetry Systems Analysis}

Table~\ref{tab:telemetry} compares telemetry technologies for student rocketry applications \cite{rf_telemetry_propagation, student_rocketry_telemetry}.

\begin{table}[h]
\centering
\caption{Telemetry System Comparison for Student Rocketry Applications}
\label{tab:telemetry}
\small
\begin{tabular}{@{}lllll@{}}
\toprule
\textbf{Technology} & \textbf{Range} & \textbf{Data Cap} & \textbf{Strength} & \textbf{Limitation} \\ \midrule
433 MHz RF & 5--10 km & Very low & Simple, long range & No ACKs \\
Beacon & 4 km/0.8 km & Low--Med & Simple & Range drops \\
LoRa SX1278 & 2--15 km & Low & Low power & Low BW \\
XBee 2.4 GHz & $<$2 km & Medium & Easy integration & Short range \\
\textbf{XBee-PRO 900HP} & \textbf{5--15 km} & \textbf{Med--High} & \textbf{Reliable} & \textbf{Higher power} \\
ESP-NOW & $<$1 km & Medium & Low latency & Not long-range \\
Cellular LTE & Global & High & Unlimited range & Network dep. \\ \bottomrule
\end{tabular}
\end{table}

The XBee-PRO 900HP provides optimal balance between range, data capacity, and protocol reliability \cite{xbee_pro_datasheet}. Its 900 MHz frequency offers superior penetration versus 2.4 GHz alternatives.

\subsection{Pyrotechnic Deployment Systems}

Nakka's Crimson Powder research provides baseline data for pyrotechnic charges \cite{nakka_crimson}. Proper formulation generates sufficient pressure while minimizing thermal output \cite{pyrotechnic_safety}. Combustion temperatures exceed 2000°C, requiring thermal isolation to prevent parachute damage \cite{modern_rocketry}.

\begin{figure}[h]
\centering
\fbox{\parbox{0.7\textwidth}{\centering\vspace{2.5cm}[Placeholder: Explosion Cap Cross-Section]\vspace{2.5cm}}}
\caption{Sealed explosion cap design showing thermal isolation and pressure channeling}
\label{fig:explosion_cap}
\end{figure}

\subsection{Electronic Control Systems}

The ESP32 platform offers dual-core architecture for parallel sensor processing and communication \cite{esp32_datasheet, embedded_flight_computers}. PWM ignition control enables precise thermal management, preventing wire burnout while enabling multiple tests \cite{pwm_ignition_systems}.

\subsection{Power and Structural Systems}

LiPo batteries provide superior energy density but require BMS for safety \cite{lipo_aerospace, bms_systems}. High-G launch loads cause battery disconnection in spring-loaded holders \cite{avionics_vibration}. Custom avionics holders provide vibration isolation and battery retention.

\begin{figure}[h]
\centering
\fbox{\parbox{0.8\textwidth}{\centering\vspace{3cm}[Placeholder: Avionics Bay Assembly]\vspace{3cm}}}
\caption{3D-printed avionics holder with integrated battery retention and vibration isolation}
\label{fig:avionics_bay}
\end{figure}

\subsection{Previous Nakuja Mission Analysis}

Table~\ref{tab:nakuja_missions} summarizes mission outcomes.

\begin{table}[h]
\centering
\caption{Nakuja Mission History and Key Findings}
\label{tab:nakuja_missions}
\begin{tabular}{@{}llll@{}}
\toprule
\textbf{Mission} & \textbf{Date} & \textbf{System} & \textbf{Key Outcome} \\ \midrule
N-1 & May 2021 & Single deploy & Baseline established \\
N-2 & Nov 2022 & Single deploy & Telemetry improved \\
N-3 & 2024 & Enhanced & Range limits identified \\
N-3.5 & 2024 & Piston deploy & Thermal damage observed \\
\textbf{N-4} & \textbf{Proposed} & \textbf{Dual deploy} & \textbf{This project} \\ \bottomrule
\end{tabular}
\end{table}

Analysis shows telemetry loss at altitudes exceeding 2km \cite{nakuja_website}. The XBee-PRO 900HP directly addresses this limitation.

\subsection{Safety Standards and Guidelines}

Amateur rocketry follows established safety protocols from organizations like Tripoli Rocketry Association \cite{tripoli_standards}. Key safety considerations include proper pyrotechnic handling \cite{pyrotechnic_safety}, adequate descent rates for safe landing \cite{descent_dynamics}, and ground station monitoring capabilities \cite{modern_rocketry}.

\begin{figure}[h]
\centering
\fbox{\parbox{0.75\textwidth}{\centering\vspace{2.5cm}[Placeholder: System Integration Diagram]\vspace{2.5cm}}}
\caption{Complete N4 recovery system showing integration of all subsystems}
\label{fig:system_integration}
\end{figure}
