\section{Literature Review}
\label{sec:review}

\subsection{Recovery Systems in Student Rocketry}

Recovery systems preserve flight hardware and recorded data \cite{nakka_recovery}. Student rocketry projects face budget constraints requiring innovative, cost-effective solutions \cite{student_rocket_design}. Dual-deployment architectures use staged recovery: drogue chute at apogee for stability, main parachute at lower altitude for soft landing \cite{dual_deploy_analysis, parachute_design}.

\begin{figure}[htbp]
\centering
\includegraphics[width=0.7\textwidth]{images/Nakuja_N3_Rocket.PNG}
\caption{Dual deployment recovery system showing drogue parachute deployment at apogee and main parachute deployment at lower altitude for controlled descent}
\label{fig:dual_deploy}
\end{figure}

The Nakuja Project's progression from N3.5 single-deployment to N4 dual-deployment architecture reflects industry evolution \cite{nakuja_website}. Previous missions identified critical improvements needed in mechanical reliability and telemetry range.

\subsection{Telemetry Systems Analysis}

Table~\ref{tab:telemetry} compares telemetry technologies for student rocketry applications \cite{rf_telemetry_propagation, student_rocketry_telemetry}.

\begin{table}[h]
\centering
\caption{Telemetry System Comparison for Student Rocketry Applications}
\label{tab:telemetry}
\small
\begin{tabular}{@{}lllll@{}}
\toprule
\textbf{Technology} & \textbf{Range} & \textbf{Data Cap} & \textbf{Strength} & \textbf{Limitation} \\ \midrule
433 MHz RF & 5--10 km & Very low & Simple, long range & No ACKs \\
Beacon & 4 km/0.8 km & Low--Med & Simple & Range drops \\
LoRa SX1278 & 2--15 km & Low & Low power & Low BW \\
XBee 2.4 GHz & $<$2 km & Medium & Easy integration & Short range \\
\textbf{XBee-PRO 900HP} & \textbf{5--15 km} & \textbf{Med--High} & \textbf{Reliable} & \textbf{Higher power} \\
ESP-NOW & $<$1 km & Medium & Low latency & Not long-range \\
Cellular LTE & Global & High & Unlimited range & Network dep. \\ \bottomrule
\end{tabular}
\end{table}

The XBee-PRO 900HP provides optimal balance between range, data capacity, and protocol reliability \cite{xbee_pro_datasheet}. Its 900 MHz frequency offers superior penetration versus 2.4 GHz alternatives.

\begin{figure}[htbp]
\centering
\includegraphics[width=0.7\textwidth]{images/Beacon_RSSI_Decay_Plot.PNG}
\caption{Beacon RSSI decay analysis showing signal degradation with distance and altitude}
\label{fig:rssi_decay}
\end{figure}

\subsection{Pyrotechnic Deployment Systems}

Nakka's Crimson Powder research provides baseline data for pyrotechnic charges \cite{nakka_crimson}. Proper formulation generates sufficient pressure while minimizing thermal output. Combustion temperatures exceed 2000°C, requiring thermal isolation to prevent parachute damage.

\begin{figure}[htbp]
\centering
\includegraphics[width=0.6\textwidth]{images/The Crimson Powder Problem_Excessive_Pop.PNG}
\caption{Excessive crimson powder deployment showing the need for precise charge control}
\label{fig:excessive_pop}
\end{figure}

\begin{figure}[htbp]
\centering
\includegraphics[width=0.55\textwidth]{images/CHarge_Holder_Melted_From_Explosion.jpeg}
\caption{Charge holder thermal damage from explosion highlighting thermal isolation requirements}
\label{fig:thermal_damage}
\end{figure}

\begin{figure}[htbp]
\centering
\includegraphics[width=0.55\textwidth]{images/Post_PopTest_Charring_And_Leak_Image.jpeg}
\caption{Post-deployment test showing charring effects and the importance of sealed containment}
\label{fig:post_test_charring}
\end{figure}

\subsection{Electronic Control Systems}

The ESP32 platform offers dual-core architecture for parallel sensor processing and communication \cite{esp32_datasheet, embedded_flight_computers}. PWM ignition control enables precise thermal management, preventing wire burnout while enabling multiple tests \cite{pwm_ignition_systems}.

\subsection{Power and Structural Systems}

LiPo batteries provide superior energy density but require BMS for safety. High-G launch loads cause battery disconnection in spring-loaded holders \cite{avionics_vibration}. Custom avionics holders provide vibration isolation and battery retention.

\begin{figure}[htbp]
\centering
\includegraphics[width=0.65\textwidth]{images/Assembled_Avionics_Bay.jpeg}
\caption{Assembled avionics bay with integrated battery retention and vibration isolation}
\label{fig:avionics_bay}
\end{figure}

\subsection{Previous Nakuja Mission Analysis}

Table~\ref{tab:nakuja_missions} summarizes mission outcomes.

\begin{table}[h]
\centering
\caption{Nakuja Mission History and Key Findings}
\label{tab:nakuja_missions}
\begin{tabular}{@{}llll@{}}
\toprule
\textbf{Mission} & \textbf{Date} & \textbf{System} & \textbf{Key Outcome} \\ \midrule
N-1 & May 2021 & Single deploy & Baseline established \\
N-2 & Nov 2022 & Single deploy & Telemetry improved \\
N-3 & 2024 & Enhanced & Range limits identified \\
N-3.5 & 2024 & Piston deploy & Thermal damage observed \\
\textbf{N-4} & \textbf{Proposed} & \textbf{Dual deploy} & \textbf{This project} \\ \bottomrule
\end{tabular}
\end{table}

Analysis shows telemetry loss at altitudes exceeding 2km \cite{nakuja_website}. The XBee-PRO 900HP directly addresses this limitation.

\begin{figure}[h]
\centering
\includegraphics[width=0.8\textwidth]{images/XBeeDiagram.png}
\caption{XBee-PRO 900HP telemetry system integration showing complete signal path}
\label{fig:xbee_integration}
\end{figure}

\subsection{Safety Standards and Guidelines}

Amateur rocketry follows established safety protocols from organizations like Tripoli Rocketry Association. Key safety considerations include proper pyrotechnic handling, adequate descent rates for safe landing, and ground station monitoring capabilities.

\begin{figure}[h]
\centering
\includegraphics[width=0.75\textwidth]{images/aduiabatic curve.png}
\caption{Adiabatic expansion analysis for parachute deployment system design}
\label{fig:adiabatic}
\end{figure}
