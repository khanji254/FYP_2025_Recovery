\section{Literature Review}
\label{sec:review}

\subsection{Coffee Berry Processing and Quality Control}

Coffee cherry quality directly influences final cup quality and market value \cite{coffee_quality_standards}. Post-harvest processing begins with sorting to separate ripe cherries (red/purple) from unripe (green), overripe (black), and defective berries \cite{coffee_processing_methods}. Sorting accuracy impacts subsequent processing steps including pulping, fermentation, and drying, with higher initial sorting quality reducing defect rates in final products.

Traditional manual sorting relies on visual inspection and manual separation, achieving 85-95\% accuracy under optimal conditions \cite{manual_coffee_sorting}. However, manual methods face scalability limitations, high labor costs, and operator fatigue-induced accuracy degradation during extended operations \cite{agricultural_labor_efficiency}.

\begin{figure}[h]
\centering
\fbox{\parbox{0.8\textwidth}{\centering\vspace{3cm}[Placeholder: Coffee Berry Classification Chart]\vspace{3cm}}}
\caption{Coffee berry classification showing ripe, unripe, overripe, and defective categories}
\label{fig:berry_classification}
\end{figure}

\subsection{Mechanical Sorting Technologies}

Mechanical coffee sorting systems employ various separation principles. Table~\ref{tab:sorting_tech} compares major sorting technologies.

\begin{table}[h]
\centering
\caption{Coffee Berry Sorting Technology Comparison}
\label{tab:sorting_tech}
\small
\begin{tabular}{@{}lllll@{}}
\toprule
\textbf{Technology} & \textbf{Accuracy} & \textbf{Throughput} & \textbf{Cost} & \textbf{Complexity} \\ \midrule
Flotation & 70--80\% & High & Low & Simple \\
Density separation & 75--85\% & Medium & Low--Med & Medium \\
Vibration table & 80--88\% & Medium & Medium & Medium \\
Mechanical actuator & 85--92\% & Med--High & Medium & High \\
\textbf{Optical sensor} & \textbf{90--98\%} & \textbf{High} & \textbf{High} & \textbf{High} \\
Air jet & 88--95\% & Very High & High & High \\ \bottomrule
\end{tabular}
\end{table}

Flotation-based systems utilize density differences between ripe and unripe berries \cite{flotation_separation}. While simple and cost-effective, accuracy limitations restrict applications to preliminary sorting stages. Density separation systems improve accuracy but require precise calibration for different coffee varieties \cite{density_sorting_systems}.

Mechanical actuator systems employ physical separation mechanisms triggered by sensor detection \cite{mechanical_sorting_actuators}. These systems offer balance between cost and performance, making them suitable for mid-scale operations \cite{small_scale_coffee_processing}.

\subsection{Sensor Technologies for Berry Classification}

Color detection represents the primary classification criterion for coffee berry maturity \cite{color_based_sorting}. RGB sensors detect color differences between ripe (red/purple) and unripe (green) berries, while more sophisticated systems employ spectral analysis for enhanced discrimination \cite{spectral_analysis_agriculture}.

\begin{figure}[h]
\centering
\fbox{\parbox{0.7\textwidth}{\centering\vspace{2.5cm}[Placeholder: Color Sensor Detection Principle]\vspace{2.5cm}}}
\caption{Color sensor detection principle showing RGB response to berry maturity levels}
\label{fig:color_sensor}
\end{figure}

Near-infrared (NIR) spectroscopy provides additional classification dimensions by analyzing internal berry composition \cite{nir_coffee_sorting}. However, NIR systems introduce significant cost increases, limiting adoption to large-scale industrial operations.

Photoelectric sensors offer cost-effective alternatives for binary classification (accept/reject) based on reflectance characteristics \cite{photoelectric_sensors_agriculture}. Integration with mechanical actuators enables real-time sorting decisions at high throughput rates \cite{automated_sorting_systems}.

\subsection{Conveyor Systems and Material Handling}

Conveyor design significantly impacts sorting system performance \cite{conveyor_design_principles}. Belt conveyors provide smooth, continuous material flow essential for optical detection systems \cite{belt_conveyor_applications}. Critical parameters include belt speed, material spacing, and vibration isolation.

Feed hopper design influences berry flow characteristics \cite{hopper_design_agricultural}. Irregular feeding patterns cause berry clustering, overwhelming sorting mechanisms and reducing accuracy \cite{material_flow_optimization}. Vibratory feeders provide controlled, uniform material presentation to sorting zones \cite{vibratory_feeding_systems}.

\begin{figure}[h]
\centering
\fbox{\parbox{0.8\textwidth}{\centering\vspace{3cm}[Placeholder: Conveyor System Design]\vspace{3cm}}}
\caption{Coffee berry conveyor system showing feed hopper, belt conveyor, and sorting mechanism}
\label{fig:conveyor_system}
\end{figure}

Belt tracking systems prevent lateral drift that misaligns berries with detection sensors \cite{belt_tracking_methods}. Spring-loaded tensioning maintains consistent belt tension across operational conditions \cite{belt_tensioning_systems}.

\subsection{Sorting Actuator Mechanisms}

Pneumatic actuators dominate high-speed sorting applications due to rapid response times (5-15 ms) \cite{pneumatic_actuators_sorting}. Air jet systems eject individual berries from product streams with precise timing \cite{air_jet_sorting}. However, pneumatic systems require compressed air infrastructure and continuous air supply \cite{pneumatic_system_design}.

Mechanical flapper actuators provide cost-effective alternatives for medium-speed operations \cite{mechanical_actuator_design}. Solenoid-driven flappers deflect berries into reject chutes based on sensor signals \cite{solenoid_applications_agriculture}. Response times of 20-40 ms limit maximum throughput but reduce infrastructure requirements.

\begin{figure}[h]
\centering
\fbox{\parbox{0.7\textwidth}{\centering\vspace{2.5cm}[Placeholder: Actuator Mechanism Diagram]\vspace{2.5cm}}}
\caption{Mechanical sorting actuator showing solenoid drive and flapper mechanism}
\label{fig:actuator_mechanism}
\end{figure}

\subsection{Control Systems and Signal Processing}

Microcontroller-based control systems enable real-time processing of sensor data and actuator coordination \cite{microcontroller_agriculture}. Arduino and ESP32 platforms offer sufficient processing power for sensor fusion and decision algorithms \cite{embedded_systems_sorting}.

Signal processing algorithms filter noise and implement classification logic \cite{signal_processing_sensors}. Threshold-based classification provides simplicity but requires calibration for different lighting conditions and berry varieties \cite{adaptive_thresholding}. Machine learning approaches improve classification accuracy through adaptive learning \cite{ml_agricultural_sorting}, though implementation complexity increases significantly.

\subsection{Performance Metrics and Optimization}

Sorting system performance evaluation employs multiple metrics. Sorting accuracy measures percentage of correctly classified berries, while throughput quantifies processing capacity (kg/hr) \cite{sorting_system_metrics}. Misclassification rates distinguish between false accepts (defects in accepted stream) and false rejects (good berries in reject stream) \cite{quality_control_metrics}.

\begin{table}[h]
\centering
\caption{Typical Sorting System Performance Benchmarks}
\label{tab:performance_metrics}
\begin{tabular}{@{}lll@{}}
\toprule
\textbf{System Type} & \textbf{Accuracy} & \textbf{Throughput (kg/hr)} \\ \midrule
Manual sorting & 85--95\% & 20--40 \\
Basic mechanical & 75--85\% & 80--120 \\
Enhanced mechanical & 85--92\% & 120--180 \\
Optical commercial & 92--98\% & 200--500 \\
Industrial optical & 95--99\% & 500--2000 \\ \bottomrule
\end{tabular}
\end{table}

Optimization strategies address trade-offs between accuracy and throughput \cite{sorting_optimization}. Reduced conveyor speed improves sensor detection time but decreases throughput. Multi-stage sorting combines rapid primary sorting with slower secondary re-sorting to balance performance \cite{multi_stage_sorting}.

\subsection{Existing Coffee Sorting Machines in Kenya}

Previous research on Kenyan coffee processing facilities identified common limitations in existing sorting equipment \cite{kenya_coffee_processing}. Mechanical wear in conveyor systems, sensor contamination from coffee pulp, and inconsistent power supply represent recurring challenges \cite{agricultural_machinery_kenya}.

Locally manufactured sorting machines demonstrate cost advantages but often compromise on component quality and sensor precision \cite{local_agricultural_equipment}. Imported commercial systems offer superior performance but face sustainability challenges due to parts availability and maintenance expertise requirements \cite{imported_machinery_challenges}.

\begin{figure}[h]
\centering
\fbox{\parbox{0.75\textwidth}{\centering\vspace{2.5cm}[Placeholder: Complete System Integration]\vspace{2.5cm}}}
\caption{Complete coffee berry sorting machine showing integrated subsystems}
\label{fig:system_integration}
\end{figure}
