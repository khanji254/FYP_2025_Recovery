\section{Design and Methodology}
\label{sec:methodology}

\subsection{Project Objectives}

\subsubsection{Main Objective}

To modify and improve the existing coffee berry sorting machine to achieve enhanced sorting accuracy (>90\%), increased throughput capacity (150-200 kg/hr), and improved mechanical reliability through systematic component upgrades and control system enhancements.

\subsubsection{Specific Objectives}

\begin{enumerate}
    \item Redesign feed hopper and vibratory feeder system for consistent, uniform berry presentation to sorting zone
    \item Upgrade conveyor belt system with improved tracking, tensioning, and speed control for optimal material flow
    \item Implement color-based optical sensor array with real-time classification algorithms for accurate ripeness detection
    \item Design and integrate mechanical actuator system (solenoid-driven flapper mechanism) with <30ms response time
    \item Develop microcontroller-based control system with sensor fusion and adaptive thresholding algorithms
    \item Implement performance monitoring system with real-time throughput and accuracy metrics
\end{enumerate}

\subsection{Design Approach}

The project employs systematic modification methodology \cite{mechanical_sorting_actuators}:

\begin{enumerate}
    \item Current System Analysis: Identify existing machine limitations and failure modes
    \item Requirements Definition: Establish performance targets for throughput and accuracy
    \item Component Selection: Specify upgraded components based on performance requirements
    \item Detailed Design: Create CAD models for mechanical modifications and circuit schematics
    \item Prototyping and Testing: Fabricate modifications and conduct validation testing
    \item Integration and Optimization: Install upgrades and tune system parameters
    \item Performance Validation: Measure sorting accuracy and throughput against targets
\end{enumerate}

\subsection{Implementation Methodology}

\subsubsection{Mechanical System Modifications}

\paragraph{Feed Hopper Redesign}
The existing feed hopper produces irregular berry flow causing clustering at the conveyor inlet. Proposed modifications include:
\begin{itemize}
    \item Enlarged hopper capacity (15 kg to 25 kg) for continuous operation
    \item Adjustable gate mechanism for flow rate control
    \item Sloped walls (45-degree angle) to prevent berry bridging \cite{hopper_design_agricultural}
    \item Vibratory feeder integration for controlled, uniform berry release
\end{itemize}

Vibratory feeder specifications: electromagnetic drive, variable frequency control (20-60 Hz), amplitude adjustment for different berry sizes \cite{vibratory_feeding_systems}.

\paragraph{Conveyor Belt System Upgrade}
Current conveyor exhibits belt tracking drift and insufficient speed control. Modifications include:
\begin{itemize}
    \item Belt tracking system: crowned pulleys and adjustable idlers \cite{belt_tracking_methods}
    \item Spring-loaded tensioning mechanism maintaining constant tension across operational conditions \cite{belt_tensioning_systems}
    \item Variable frequency drive (VFD) for precise speed control (0.3-1.2 m/s range)
    \item Food-grade PVC belt material with textured surface for berry positioning
\end{itemize}

\paragraph{Sorting Actuator Mechanism}
New mechanical actuator system employs solenoid-driven flapper design:
\begin{itemize}
    \item 12V DC push-type solenoids (25mm stroke, 10N force) \cite{solenoid_applications_agriculture}
    \item Lightweight aluminum flapper arms (100mm length) for rapid actuation
    \item Adjustable mounting brackets for precise positioning relative to sensor detection zone
    \item Modular design enabling individual actuator replacement without system disassembly
\end{itemize}

\begin{figure}[h]
\centering
\fbox{\parbox{0.7\textwidth}{\centering\vspace{2.5cm}[Placeholder: Mechanical Assembly CAD Model]\vspace{2.5cm}}}
\caption{CAD model showing upgraded conveyor system, feed hopper, and sorting actuators}
\label{fig:mechanical_assembly}
\end{figure}

\subsubsection{Sensor System Implementation}

\paragraph{Optical Sensor Array}
Color-based detection employs RGB sensors positioned above conveyor belt:
\begin{itemize}
    \item TCS34725 RGB color sensors (4 units distributed across belt width) \cite{color_based_sorting}
    \item LED illumination array (5500K white LEDs) for consistent lighting conditions
    \item Sensor height: 50mm above belt surface for optimal detection zone
    \item Detection resolution: individual berry identification at 0.8 m/s belt speed
\end{itemize}

Sensor calibration procedure:
\begin{enumerate}
    \item Establish color thresholds for ripe (red/purple), unripe (green), and overripe (black) berries
    \item Generate lookup tables mapping RGB values to ripeness classifications
    \item Implement ambient light compensation algorithms \cite{adaptive_thresholding}
    \item Validate classification accuracy with known berry samples
\end{enumerate}

\paragraph{Position Sensing}
Photoelectric sensors detect berry presence and trigger position tracking:
\begin{itemize}
    \item Through-beam photoelectric sensors at detection zone entrance
    \item Encoder feedback from conveyor motor for position tracking
    \item Time-of-flight calculation synchronizing detection with actuator firing
\end{itemize}

\begin{figure}[h]
\centering
\fbox{\parbox{0.7\textwidth}{\centering\vspace{2.5cm}[Placeholder: Sensor Positioning Diagram]\vspace{2.5cm}}}
\caption{Sensor array positioning showing RGB sensors, LED illumination, and photoelectric sensors}
\label{fig:sensor_positioning}
\end{figure}

\subsubsection{Control System Architecture}

\paragraph{Microcontroller Platform}
ESP32 development board provides control system processing:
\begin{itemize}
    \item Dual-core architecture: Core 0 for sensor processing, Core 1 for actuator control \cite{esp32_datasheet}
    \item I2C bus communication with RGB sensors
    \item GPIO outputs for solenoid driver control
    \item WiFi connectivity for remote monitoring and parameter adjustment
\end{itemize}

\paragraph{Signal Processing and Classification}
Real-time classification algorithm:
\begin{enumerate}
    \item RGB sensor data acquisition at 10 Hz sampling rate
    \item Digital filtering (moving average, 5-point window) for noise reduction \cite{signal_processing_sensors}
    \item Threshold comparison: classify as ACCEPT if R $>$ 120 and G $<$ 100, otherwise REJECT
    \item Position tracking: calculate actuator firing delay based on belt speed and sensor-to-actuator distance
    \item Actuator command: trigger solenoid pulse (50ms duration) at calculated delay
\end{enumerate}

\paragraph{Power System}
\begin{itemize}
    \item Primary power: 220V AC mains via transformer and rectifier
    \item 12V DC rail: solenoid actuators and sensors (regulated supply, 5A capacity)
    \item 5V DC rail: ESP32 and signal electronics (buck converter, 2A capacity)
    \item Emergency stop circuit: hardwired mechanical switch disconnecting solenoid power
\end{itemize}

\begin{figure}[h]
\centering
\fbox{\parbox{0.7\textwidth}{\centering\vspace{2.5cm}[Placeholder: Control System Block Diagram]\vspace{2.5cm}}}
\caption{Control system architecture showing sensor inputs, microcontroller processing, and actuator outputs}
\label{fig:control_system}
\end{figure}

\subsection{Testing and Validation Plan}

\subsubsection{Component-Level Testing}
\begin{itemize}
    \item Vibratory feeder: characterize flow rate vs. frequency and amplitude settings
    \item Conveyor belt: verify tracking stability across speed range (0.3-1.2 m/s)
    \item RGB sensors: calibrate color thresholds with sample berries under various lighting conditions
    \item Solenoid actuators: measure response time and force output, verify actuation reliability (1000+ cycle test)
    \item Control algorithms: validate classification accuracy with labeled berry dataset (300+ samples)
\end{itemize}

\subsubsection{System Integration Testing}
\begin{itemize}
    \item Feed system: verify uniform berry spacing on conveyor belt (visual inspection and high-speed video)
    \item Sensor-actuator synchronization: confirm timing accuracy by testing with marked berries
    \item Throughput measurement: process known berry quantities and measure completion time
    \item Accuracy measurement: manually verify sorted berry classification (100 berry sample minimum)
\end{itemize}

\subsubsection{Performance Validation Protocol}
Comprehensive testing procedure:
\begin{enumerate}
    \item Prepare test batches: 10 kg mixed berries (known distribution of ripe/unripe/defect)
    \item Process batches at target operating speed (0.8 m/s belt speed)
    \item Collect and weigh both accepted and rejected streams
    \item Manually inspect 5\% random sample from each stream
    \item Calculate metrics: sorting accuracy, false accept rate, false reject rate, throughput (kg/hr)
    \item Repeat testing for 5 batches to establish performance consistency
    \item Document failure modes and optimization opportunities
\end{enumerate}

Target performance metrics:
\begin{itemize}
    \item Sorting accuracy: $>$ 90\% (correctly classified berries / total berries)
    \item Throughput: 150-200 kg/hr (2.5x improvement over baseline 50-75 kg/hr)
    \item False accept rate: $<$ 8\% (defects in accepted stream)
    \item False reject rate: $<$ 12\% (good berries in rejected stream)
    \item System uptime: $>$ 95\% during 4-hour continuous operation test
\end{itemize}
