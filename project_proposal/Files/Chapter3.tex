\section{Design and Methodology}
\label{sec:methodology}

\subsection{Main Objectives}

The project aims to design and fabricate a reliable dual recovery system for the N4 rocket with extended telemetry range. The following four objectives encapsulate the technical requirements:

\begin{enumerate}
    \item \textbf{Long-Range Telemetry Implementation:} Implement Digi XBee-PRO 900HP telemetry system for robust long-range data transmission exceeding 5km with less than 5\% packet loss, enabling real-time monitoring throughout the flight envelope.
    
    \item \textbf{Pyrotechnic Deployment System:} Design and validate PWM-controlled ignition system for regulating nichrome wire temperature, coupled with sealed explosion cap design for consistent and reliable parachute deployment at apogee and main deployment altitudes.
    
    \item \textbf{Comprehensive Monitoring and Safety:} Develop comprehensive feedback monitoring system for battery voltage, deployment pin continuity, and system health diagnostics with launch inhibit functionality activating below critical voltage thresholds.
    
    \item \textbf{Ruggedized Mechanical Integration:} Design and fabricate ruggedized 3D-printed PETG avionics holder with vibration isolation, battery retention, and center-of-gravity optimization to withstand high-G launch loads while maintaining structural integrity.
\end{enumerate}

\subsection{Design Approach}

The project employs systematic design-build-test methodology \cite{student_rocket_design}:

\begin{enumerate}
    \item Requirements Analysis: Define system specifications based on N4 mission profile
    \item Conceptual Design: Develop solution architectures
    \item Detailed Design: Create CAD models and circuit schematics
    \item Prototyping: Fabricate components
    \item Testing and Validation: Conduct ground and flight tests
    \item Iteration: Refine designs based on results
\end{enumerate}

\subsection{Implementation Methodology}

\subsubsection{Mechanical System Design}

\paragraph{Sealed Explosion Cap}
Sealed explosion cap houses Crimson Powder pyrotechnic charge, containing combustion products while directing pressure for deployment \cite{nakka_crimson}. High-temperature resistant materials withstand combustion while enabling multiple test cycles.

\paragraph{Avionics Bay Structure}
Custom 3D-printed PETG holder secures flight computer, LiPo batteries, and BMS with vibration isolation \cite{avionics_vibration}. Design maintains center of gravity stability under high-G loads (see current design in Appendix~\ref{fig:current_fc_holder}).



\subsubsection{Electrical System Architecture}

\paragraph{Hardware Components}
ESP32 dual-core microcontroller processes telemetry data \cite{esp32_datasheet}. XBee-PRO 900HP module provides 5+km range at 900 MHz \cite{xbee_pro_datasheet}. Custom PCB eliminates loose connections \cite{embedded_flight_computers}. Current circuit implementation is detailed in Appendix~\ref{fig:current_circuit}.

\begin{figure}[htbp]
\centering
\includegraphics[width=0.7\textwidth]{images/XBeeDiagram.png}
\caption{Proposed XBee-PRO 900HP telemetry system architecture }
\label{fig:xbee_proposed}
\end{figure}

\paragraph{PWM Ignition Control}
MOSFET-based PWM driver controls nichrome wire heating, preventing burnout while enabling reusable testing \cite{pwm_ignition_systems}.

\begin{figure}[htbp]
\centering
\includegraphics[width=0.5\textwidth]{images/Cut_Nichrome_Wire.PNG}
\caption{Nichrome wire demonstration showing thermal melting and cutting mechanism - illustrating the one-shot deployment characteristic}
\label{fig:nichrome_wire}
\end{figure}

\paragraph{Power Management}
BMS monitors LiPo battery health \cite{lipo_aerospace, bms_systems}. Real-time feedback transmits battery voltage and igniter continuity. Launch inhibit activates below 14.0V threshold.



\subsection{Testing and Validation Plan}

\subsubsection{Ground Testing Protocol}
\begin{itemize}
    \item Telemetry range testing at 5+ km distances
    \item PWM ignition characterization with thermal imaging
    \item Explosion cap deployment force measurements
    \item Vibration testing on shake table (15G+)
    \item Full system integration verification
\end{itemize}

\subsubsection{Flight Testing Protocol}
Progressive testing: low-altitude validation flights before full mission profile.
