\section{Design and Methodology}
\label{sec:methodology}

\subsection{Design Approach}

The project employs systematic design-build-test methodology \cite{student_rocket_design}:

\begin{enumerate}
    \item Requirements Analysis: Define system specifications based on N4 mission profile
    \item Conceptual Design: Develop solution architectures
    \item Detailed Design: Create CAD models and circuit schematics
    \item Prototyping: Fabricate components
    \item Testing and Validation: Conduct ground and flight tests
    \item Iteration: Refine designs based on results
\end{enumerate}

\subsection{Implementation Methodology}

\subsubsection{Mechanical System Design}

\paragraph{Sealed Explosion Cap}
Sealed explosion cap houses Crimson Powder pyrotechnic charge, containing combustion products while directing pressure for deployment \cite{nakka_crimson}. High-temperature resistant materials withstand combustion while enabling multiple test cycles.

\paragraph{Avionics Bay Structure}
Custom 3D-printed PETG holder secures flight computer, LiPo batteries, and BMS with vibration isolation \cite{avionics_vibration}. Design maintains center of gravity stability under high-G loads (see current design in Appendix~\ref{fig:current_fc_holder}).

Figure~\ref{fig:cad_holder} shows the CAD rendering of the avionics bay assembly with precise mounting features and component spacing requirements.

\begin{figure}[htbp]
\centering
\includegraphics[width=0.65\textwidth]{images/Current_CAD.PNG}
\caption{CAD rendering of the avionics bay assembly showing component layout, mounting features, and dimensional constraints}
\label{fig:cad_holder}
\end{figure}



\subsubsection{Electrical System Architecture}

\paragraph{Hardware Components}
ESP32 dual-core microcontroller processes telemetry data \cite{esp32_datasheet}. XBee-PRO 900HP module provides 5+km range at 900 MHz \cite{xbee_pro_datasheet}. Custom PCB eliminates loose connections \cite{embedded_flight_computers}. Current circuit implementation is detailed in Appendix~\ref{fig:current_circuit}.

\begin{figure}[htbp]
\centering
\includegraphics[width=0.7\textwidth]{images/XBeeDiagram.png}
\caption{Proposed XBee-PRO 900HP telemetry system architecture }
\label{fig:xbee_proposed}
\end{figure}

\paragraph{PWM Ignition Control}
MOSFET-based PWM driver controls nichrome wire heating, preventing burnout while enabling reusable testing \cite{pwm_ignition_systems}.

\begin{figure}[htbp]
\centering
\includegraphics[width=0.5\textwidth]{images/Cut_Nichrome_Wire.PNG}
\caption{Nichrome wire demonstration showing thermal melting and cutting mechanism - illustrating the one-shot deployment characteristic}
\label{fig:nichrome_wire}
\end{figure}

\paragraph{Power Management}
BMS monitors LiPo battery health \cite{lipo_aerospace, bms_systems}. Real-time feedback transmits battery voltage and igniter continuity. Launch inhibit activates below 14.0V threshold.

The battery management system employs an ADS1115 16-bit analog-to-digital converter for precise voltage and chute state monitoring. Battery voltage is acquired through a 47k$\Omega$ / 10k$\Omega$ voltage divider network providing a measurement range of 12V to 16.8V. The voltage measurement formula is calibrated as: $V_{battery} = V_{ADC} \times \frac{57.0}{10.0}$.

Chute detection utilizes dual analog inputs (channels 1 and 2) to monitor sensor states. Signals exceeding 1000 counts indicate deployment readiness. The ADS1115 is configured with ±4.096V full-scale range gain for optimal resolution across all measured channels.

\subsubsection{Electrical Design Concepts and Simulations}

Figure~\ref{fig:base_station_design} presents the conceptual design of the ground base station RF receiver architecture, integrating XBee-PRO telemetry with local data logging.

\begin{figure}[htbp]
\centering
\includegraphics[width=0.75\textwidth]{images/Base_Station_Design.jpeg}
\caption{Base station receiver circuit concept showing XBee-PRO 900HP integration, RF filtering, and ground control processing elements}
\label{fig:base_station_design}
\end{figure}

Figure~\ref{fig:rocket_xbee_extension} shows the XBee extension circuit design for the rocket avionics bay, demonstrating signal routing and antenna connection topology.

\begin{figure}[htbp]
\centering
\includegraphics[width=0.75\textwidth]{images/Rocket_Extension_Design.jpeg}
\caption{Rocket XBee extension circuit pinout and Fritzing layout illustrating power conditioning, antenna interface, and signal routing}
\label{fig:rocket_xbee_extension}
\end{figure}

Figure~\ref{fig:bms_proteus_design} illustrates the detailed Proteus schematic and simulation of the battery management system, showing the ADS1115 ADC configuration, voltage divider network, chute detection circuits, and power distribution.

\begin{figure}[htbp]
\centering
\includegraphics[width=0.85\textwidth]{images/Battery _Monitoring_Proteus_Design.PNG}
\caption{Battery Management System Proteus design showing ADS1115 ADC with voltage divider (47k$\Omega$/10k$\Omega$), dual chute detection inputs, and power distribution topology}
\label{fig:bms_proteus_design}
\end{figure}



\subsection{Testing and Validation Plan}

\subsubsection{Ground Testing Protocol}
\begin{itemize}
    \item Telemetry range testing at 5+ km distances
    \item PWM ignition characterization with thermal imaging
    \item Explosion cap deployment force measurements
    \item Vibration testing on shake table (15G+)
    \item Full system integration verification
\end{itemize}

\subsubsection{Flight Testing Protocol}
Progressive testing: low-altitude validation flights before full mission profile.
