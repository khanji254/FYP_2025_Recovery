\section{Conclusion}
\label{sec:conclusion}

The proposed modifications and improvements to the coffee berry sorting machine address critical operational deficiencies through systematic upgrade of mechanical, sensor, and control subsystems. The implementation of an enhanced feed hopper and vibratory feeder system will eliminate irregular berry flow patterns that currently degrade sorting accuracy. Conveyor belt system upgrades, including belt tracking improvements and variable frequency drive integration, will provide consistent material handling essential for reliable sorting performance.

The integration of RGB color sensor arrays with adaptive classification algorithms represents a significant technological advancement over the existing system's rudimentary detection mechanism. Solenoid-driven mechanical actuators with <30ms response times will enable high-throughput operation while maintaining classification accuracy targets exceeding 90\%. The ESP32-based control system provides sophisticated signal processing and actuator coordination capabilities previously unavailable in the existing machine design.

This project addresses a practical agricultural engineering challenge with immediate economic impact for coffee processing operations. Enhanced sorting efficiency directly translates to reduced labor costs, improved product quality, and increased facility throughput. The modifications employ cost-effective technologies appropriate for mid-scale agricultural processing facilities in Kenya, ensuring accessibility and sustainability beyond this specific implementation.

The design methodology and technical solutions developed through this project provide replicable frameworks applicable to similar sorting challenges across agricultural processing sectors. Successful completion will demonstrate the viability of semi-automated sorting system upgrades for resource-constrained agricultural facilities, contributing to the broader adoption of precision agriculture technologies in East Africa.

Project documentation, including validated design specifications, sensor calibration procedures, and performance characterization data, will serve as valuable resources for agricultural engineering education and small-scale equipment manufacturers. The hands-on experience gained through mechatronic system integration, sensor technology application, and performance optimization provides essential practical engineering skills complementing theoretical coursework.
