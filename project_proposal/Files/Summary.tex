\section{Conclusion}
\label{sec:conclusion}

The proposed dual recovery system for the N4 rocket addresses critical gaps in current student rocketry recovery systems through comprehensive integration of advanced telemetry, controlled deployment mechanisms, and real-time monitoring capabilities. The implementation of XBee-PRO 900HP telemetry eliminates range limitations that have historically compromised mission success, while PWM-controlled ignition enables reusable deployment testing and eliminates thermal damage risks to recovery systems.

The sealed explosion cap design represents a significant advancement in deployment reliability, ensuring consistent performance while protecting critical parachute components. Integration of real-time feedback systems for battery voltage and igniter continuity provides ground operators with unprecedented visibility into system status, enabling proactive identification of potential failure modes before launch operations.

This project establishes a foundation for future high-power rocketry missions within the Nakuja project and contributes valuable empirical data regarding charge-to-ejection force relationships and ruggedized avionics design for student rocket applications. The modular system architecture enables future enhancements and adaptations for subsequent missions while maintaining core reliability improvements achieved through this development effort.

Successful completion of this project will significantly improve mission success probability for the N4 rocket while providing a validated design reference for future student rocketry recovery system development. The lessons learned and documentation generated will serve as valuable resources for the Nakuja Project and the broader student rocketry community.
