\documentclass[12pt,a4paper]{article}

% Packages
\usepackage[utf8]{inputenc}
\usepackage{graphicx}        % For images
\usepackage{caption}         % For captions
\usepackage{subcaption}      % For subfigures
\usepackage{float}           % For precise figure placement
\usepackage[hidelinks]{hyperref} % For clickable references
\usepackage{cite}            % For citations

% Set graphics path (optional - if images are in a subfolder)
\graphicspath{{images/}}

% Document information
\title{Sample LaTeX Document with Images and References}
\author{Your Name}
\date{\today}

\begin{document}

\maketitle

\begin{abstract}
This is a sample LaTeX document demonstrating how to include images, create figures with captions, and use references. It shows various ways to insert and position images in your document.
\end{abstract}

\section{Introduction}

This document demonstrates the basics of working with images in LaTeX. You can include images in various formats such as PNG, JPG, and PDF \cite{latex2e}.

\section{Including a Single Image}

To include an image, use the \texttt{figure} environment. Figure \ref{fig:example1} shows a simple example.

\begin{figure}[H]
    \centering
    \includegraphics[width=0.5\textwidth]{example-image.jpg}
    \caption{This is an example image with caption}
    \label{fig:example1}
\end{figure}

As seen in Figure \ref{fig:example1}, images can be referenced throughout the text using labels.

\section{Controlling Image Size}

You can control image size in several ways:

\begin{itemize}
    \item By width: \texttt{width=0.5\textbackslash textwidth}
    \item By height: \texttt{height=5cm}
    \item By scale: \texttt{scale=0.8}
\end{itemize}

\begin{figure}[H]
    \centering
    \includegraphics[width=0.3\textwidth]{example-image.jpg}
    \caption{Smaller image (30\% of text width)}
    \label{fig:small}
\end{figure}

\section{Multiple Images Side by Side}

You can place multiple images side by side using the \texttt{subcaption} package (see Figure \ref{fig:multiple}).

\begin{figure}[H]
    \centering
    \begin{subfigure}{0.45\textwidth}
        \centering
        \includegraphics[width=\textwidth]{example-image-a.jpg}
        \caption{First image}
        \label{fig:sub1}
    \end{subfigure}
    \hfill
    \begin{subfigure}{0.45\textwidth}
        \centering
        \includegraphics[width=\textwidth]{example-image-b.jpg}
        \caption{Second image}
        \label{fig:sub2}
    \end{subfigure}
    \caption{Two images side by side}
    \label{fig:multiple}
\end{figure}

You can reference individual subfigures: Figure \ref{fig:sub1} and Figure \ref{fig:sub2}.

\section{Image Positioning}

The \texttt{[H]} option forces the figure to appear exactly where you place it. Other options include:

\begin{itemize}
    \item \texttt{[h]} - approximately here
    \item \texttt{[t]} - top of page
    \item \texttt{[b]} - bottom of page
    \item \texttt{[p]} - separate page
    \item \texttt{[H]} - exactly here (requires float package)
\end{itemize}

\section{References and Citations}

LaTeX makes it easy to manage citations using BibTeX \cite{knuth1984texbook}. You can cite multiple sources at once \cite{latex2e, knuth1984texbook}.

\section{Conclusion}

This document demonstrated the basics of including images and using references in LaTeX. For more information, consult the documentation \cite{latex2e}.

% Bibliography
\bibliographystyle{plain}
\bibliography{references}

\end{document}
